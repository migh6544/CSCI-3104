\documentclass[11pt]{article} 
\usepackage[english]{babel}
\usepackage[utf8]{inputenc}
\usepackage[margin=0.5in]{geometry}
\usepackage{amsmath}
\usepackage{amsthm}
\usepackage{amsfonts}
\usepackage{amssymb}
\usepackage[usenames,dvipsnames]{xcolor}
\usepackage{graphicx}
\usepackage[siunitx]{circuitikz}
\usepackage{tikz}
\usetikzlibrary{calc,arrows.meta}
\usepackage[colorinlistoftodos, color=orange!50]{todonotes}
\usepackage{hyperref}
\usepackage[numbers, square]{natbib}
\usepackage{fancybox}
\usepackage{epsfig}
\usepackage{soul}
\usepackage[framemethod=tikz]{mdframed}
\usepackage[shortlabels]{enumitem}
\usepackage[version=4]{mhchem}
\usepackage{multicol}
\usepackage{mathtools}
\usepackage{comment}
\usepackage{enumitem}
\usepackage[utf8]{inputenc}
\usepackage{listings}
\usepackage{color}
\usepackage[numbers]{natbib}
\usepackage{subfiles}
\usepackage{tkz-berge}
\usepackage{algorithm}
\usepackage[noend]{algpseudocode}


\newtheorem{prop}{Proposition}[section]
\newtheorem{thm}{Theorem}[section]
\newtheorem{lemma}{Lemma}[section]
\newtheorem{cor}{Corollary}[prop]

\theoremstyle{definition}
\newtheorem{definition}{Definition}

\theoremstyle{definition}
\newtheorem{required}{Problem}

\theoremstyle{definition}
\newtheorem{ex}{Example}

\tikzset{
	vertex/.style={circle,draw,minimum size=16, inner sep=0pt,font=\normalsize},
	every node/.style={draw=none,rectangle,font=\scriptsize,outer sep=0pt,inner sep=2pt},
	directed/.style={arrows={-Stealth[length=7pt]},font=\small},
	caption/.style={text width=6cm,align=center,rectangle,draw}
}


\setlength{\marginparwidth}{3.4cm}
%#########################################################

%To use symbols for footnotes
\renewcommand*{\thefootnote}{\fnsymbol{footnote}}
%To change footnotes back to numbers uncomment the following line
%\renewcommand*{\thefootnote}{\arabic{footnote}}

% Enable this command to adjust line spacing for inline math equations.
% \everymath{\displaystyle}

% _______ _____ _______ _      ______ 
%|__   __|_   _|__   __| |    |  ____|
%   | |    | |    | |  | |    | |__   
%   | |    | |    | |  | |    |  __|  
%   | |   _| |_   | |  | |____| |____ 
%   |_|  |_____|  |_|  |______|______|
%%%%%%%%%%%%%%%%%%%%%%%%%%%%%%%%%%%%%%%

\title{
\normalfont \normalsize 
\textsc{CSCI 3104 Fall 2021 \\ 
Instructors: Profs. Grochow and Waggoner} \\
[10pt] 
\rule{\linewidth}{0.5pt} \\[6pt] 
\huge Quiz- Standard 18 \\
\rule{\linewidth}{2pt}  \\[10pt]
}
%\author{Your Name}
\date{}

\begin{document}
\definecolor {processblue}{cmyk}{0.96,0,0,0}
\definecolor{processred}{rgb}{200, 0, 0}
\definecolor{processgreen}{rgb}{0, 255, 0}
\DeclareGraphicsExtensions{.png}
\DeclareGraphicsExtensions{.gif}
\DeclareGraphicsExtensions{.jpg}

\maketitle


%%%%%%%%%%%%%%%%%%%%%%%%%
%%%%%%%%%%%%%%%%%%%%%%%%%%
%%%%%%%%%%FILL IN YOUR NAME%%%%%%%
%%%%%%%%%%AND STUDENT ID%%%%%%%%
%%%%%%%%%%%%%%%%%%%%%%%%%%
\noindent
Due Date \dotfill October / $31^{st}$ / 2021 \\
Name \dotfill \textbf{Michael Ghattas} \\
Student ID \dotfill \textbf{109200649} \\


\tableofcontents

\section{Instructions}
 \begin{itemize}
	\item The solutions \textbf{should be typed}, using proper mathematical notation. We cannot accept hand-written solutions. \href{http://ece.uprm.edu/~caceros/latex/introduction.pdf}{Here's a short intro to \LaTeX.}
	\item You should submit your work through the \textbf{class Canvas page} only. Please submit one PDF file, compiled using this \LaTeX \ template.
	\item You may not need a full page for your solutions; pagebreaks are there to help Gradescope automatically find where each problem is. Even if you do not attempt every problem, please submit this document with no fewer pages than the blank template (or Gradescope has issues with it).

	\item You \textbf{may not collaborate with other students}. \textbf{Copying from any source is an Honor Code violation. Furthermore, all submissions must be in your own words and reflect your understanding of the material.} If there is any confusion about this policy, it is your responsibility to clarify before the due date. 

	\item Posting to \textbf{any} service including, but not limited to Chegg, Discord, Reddit, StackExchange, etc., for help on an assignment is a violation of the Honor Code.

	\item You \textbf{must} virtually sign the Honor Code (see Section \ref{HonorCode}). Failure to do so will result in your assignment not being graded.
\end{itemize}


\section{Honor Code (Make Sure to Virtually Sign)} \label{HonorCode}

\begin{required}
\noindent 
\begin{itemize}
\item My submission is in my own words and reflects my understanding of the material.
\item I have not collaborated with any other person.
\item I have not posted to external services including, but not limited to Chegg, Discord, Reddit, StackExchange, etc.
\item I have neither copied nor provided others solutions they can copy.
\end{itemize}

%\noindent In the specified region below, clearly indicate that you have upheld the Honor Code. Then type your name. 
\end{required}

\begin{proof}[I agree to the above, Michael Ghattas.]
%% Typing "I agree to the above," followed by your name is sufficient.
\end{proof}


\newpage
\section{Standard 18: Divide and Conquer.}

\begin{required}
Given an array $A[1, ..., n]$, we say it contains a duplicate if there are two distinct indices $i \neq j$ such that $A[i]=A[j]$. Consider the following divide and conquer algorithm for counting the number of duplicates.

\begin{verbatim}
countDuplicates (A[1, ..., n], integer p, integer q): 
    if length(A) == 2 {
        if A[0] == A[1] {
            return 1
        }
        else {
            return 0
        }
    }
    if q > p {
        r = floor ((p+q)/2)
        L = countDuplicates (A, p, r)
        R = countDuplicates (A, r+1, q)
        return L + R
    }
    else {
        return 0
    }
\end{verbatim}

\noindent Will the above algorithm return the correct number of duplicates? \textbf{Explain and justify} your answer by computing \texttt{countDuplicates(A, 1, n)} and showing what the algorithm does at each step. If the algorithm does not correctly count duplicates, give an example to illustrate that the algorithm fails.
\end{required}


\begin{proof}[Answer:] \
\item \textbf{First it is important to note the ambiguity of what integers $p$ and $q$ represent! It has not been specified that they represent the $Start / Smallest$ and $End / Largest$ values in the array.}
\item \textbf{First we will assume the intent here is integer $p$ represents the $Smallest$ value, while integer $q$ represents the $Largest$ value in the array.}
\item \textbf{We proceed with our example where $A[3, 6, 6, 3]$, $p = 3$, and $q = 6$.}
\begin{itemize}
\item Since $A[0] != A[1]$, the function returns 0 and proceeds to the next $if$ statement
\item $(q = 6) > (p = 3) \to r = floor \frac{3 + 6}{2} = 4$
\item $L = countDuplicates (A, 3, 4)$ is called
\item Since $A[0] != A[1]$, the function returns 0 and proceeds to the next $if$ statement
\item $(q = 4) > (p = 3) \to r = floor \frac{3 + 4}{2} = 3$
\item $L = countDuplicates (A, 3, 3)$ is called
\item Since $A[0] = A[1]$, the function returns 1and proceeds to the next $if$ statement
\item $R = countDuplicates (A, (4+1), 6)$ is called
\item Since $A[0] != A[1]$, the function returns 0 and proceeds to the next $if$ statement
\item $(q = 6) > (p = 5) \to r = floor \frac{5 + 6}{2} = 5$
\item $R = countDuplicates (A, (5+1), 6)$ is called
\item Since $A[0] = A[1]$, the function returns 1and proceeds to the next $if$ statement
\item The function return $L+R = 2$ 
\item The function returns 0 and terminates successfully identifying 2 duplicates \\

\item \textbf{Now we will assume the intent here is integer $p$ represents the $Start$ value, while integer $q$ represents the $End$ value in the array.}
\item \textbf{Now we proceed with our example where $A[3, 6, 6, 3]$, $p = 3$, and $q = 6$.}
\item Since $A[0] != A[1]$, the function returns 0 and proceeds to the next $if$ statement
\item $(q = 3) > (p = 3) \to r = floor \frac{3 + 3}{2} = 3$
\item $L = countDuplicates (A, 3, 3)$ is called
\item Since $A[0] = A[1]$, the function returns 1and proceeds to the next $if$ statement
\item $R = countDuplicates (A, (3+1), 3)$ is called
\item Since $A[0] != A[1]$, the function returns 0 and proceeds to the next $if$ statement
\item $(q = 3) < (p = 4)$
\item The function return $L+R = 1$ 
\item The function returns 0 and terminates unsuccessfully identifying only 1 duplicate \\
\end{itemize}
\item \textbf{Thus we conclude that the function does not always work as intended and \color{red} the algorithm fails.}

% YOUR ANSWER HERE
\end{proof}

%Include an Image: \includegraphics{ImageFileName}
%Include an Image and Rotate 90 degree: \includegraphics[angle=90]{ImageFileName}
%Include an Image, Rotate by 180 degrees, and scale by 50\% \includegraphics[scale=0.5, angle=90]{ImageFileName}



%%%%%%%%%%%%%%%%%%%%%%%%%%%%%%%%%%%%%%%%%%%%%%%%%%
\end{document} % NOTHING AFTER THIS LINE IS PART OF THE DOCUMENT



